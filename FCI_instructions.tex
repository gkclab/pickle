\documentclass{article}
\usepackage[utf8]{inputenc}
\usepackage{hyperref}

\title{Full CI starter project}
\date{}

\begin{document}

\maketitle

This is a brief tutorial to write a full configuration interaction (FCI) code.

\vspace{1cm}

Originally written by Qiming Sun.

\vspace{1cm}

The FCI problem can be solved step by step:

\begin{enumerate}

\item Get familiar with Python, and the numpy/scipy libraries. If you have no experience, 
      you can read this numpy doc
      \url{https://docs.scipy.org/doc/numpy/reference/}.

      If you need install Python and all the relevant libraries, we highly recommend
      the \href{https://www.anaconda.com/distribution/}{Anaconda package}.

      \textbf{Time:} $\sim$ 1 week if you don't have any experience with Python

\item Use the Slater-Condon rule to evaluate the Hamiltonian in the CI basis,
      then directly diagonalize it using numpy.  You can use the attached
      1-electron and 2-electron integrals (for H6 chain, STO-3G basis, 6
      orbital, 6 electrons) to do this step (the files are called h1e.npy and h2e.npy).  
      The reference energy is $-7.8399080148963369$.
      
      \textbf{Time:} $\sim$ 2 weeks


\item  Install PySCF and go through the tutorial \url{https://sunqm.github.io/pyscf/tutorial.html}
       to get familiar with the package, so that you can access the integrals
       for other systems as you want.

       \textbf{Time:} $<$ 1 week


\item   For a faster and more technical implementation, you can move on to the 
        so-called ``string-based"
        determinant-CI (or direct-CI) algorithm.  It is more of a challenge.

\item   Read about the Davidson algorithm for matrix diagonalization
        and implement it for any symmetric matrix, e.g. the Hamiltonian you 
        built up in step 2 for the H6 system.

        \textbf{Time:} this may take time, perhaps 1 to 2 weeks

\item  Read Knowles and Handy's paper (1984).  This is the most difficult step in
       this project.  You may need to read a lot of relevant literature to get
       familiar with concepts like second quantization, the physical vacuum vs
       HF vacuum, etc.
       
       \textbf{Time:} perhaps 2 - 3 weeks or more

\item  To program direct-CI, you should first implement functions to handle
       the FCI string, then the H*C operation.  Your original implementation 
       using numpy diagonalization
       will be a great aid in debugging the program.  

       \textbf{Time:} $\sim$ 1 week once you
       fully understand step 6.

\item  Use the FCI strings to evaluate the Hamiltonian matrix elements. This
       offers a good check for your step 2 and step 7.
       
       \textbf{Time:}  ~ 3 days

\item  Put everything together: the Davidson diagonalization solver,
       preconditioner (can be taken from your program in the previsou step), H*C
       operation, and debug and tests.

       \textbf{Time:} ~ 3 days.

\end{enumerate}

Congratulations! The FCI project is finished!  This project
should take 5 to 10 weeks in total if progress moves smoothly as
expected.  If your programs are completely written in Python, it should
be able to solve problem with a maximum system size of $\sim$ 12 orbitals 
with 12 electrons.  If you have
interest in extending the code to larger systems, you have some different
possibilities.

One is to optimize the FCI code, e.g. using Cython to compile some piece
of your code, or rewrite some code in C to get the best
performance.  If you just want to spend a little time to understand how
this optimization can happen, you can consider reading the FCI module in
PySCF which is a very efficient implementation.

The second choice is to read and implement some flavor of select-CI from recent work.
This might be able to handle $\sim$ 5 more orbitals based on your Python FCI
program.


\end{document}
